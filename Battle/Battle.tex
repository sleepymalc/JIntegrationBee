\documentclass[12pt, t]{beamer}
\usepackage{graphicx}
\usepackage{amsmath}
\usepackage{setspace}
\usepackage{float} 
\usepackage{multido}
\usepackage{multirow}
\usepackage{array}
\usepackage{enumerate}
\usepackage{booktabs}
\usepackage{indentfirst} 
\usepackage[style=mla]{biblatex}
\usepackage{subcaption}
\usepackage{hyperref}
\usepackage{textpos}
\usepackage{mathtools, nccmath}

\makeatletter
\let\@@magyar@captionfix\relax
\makeatother

\definecolor{Turquoise3}{RGB}{0, 134, 139}
\renewcommand{\emph}[1]{{\color{Turquoise3}\textsl{#1}}}
\newcommand{\C}{\mathbb{C}} \newcommand{\F}{\mathbb{F}} \newcommand{\R}{\mathbb{R}} \newcommand{\Q}{\mathbb{Q}}
\newcommand{\N}{\mathbb{N}}
\newcommand{\myseries}[2]{$#1_1,#1_2,\dots,#1_#2$}
\newcommand{\nullspace}{~\\[15pt]}
\newcommand{\Remark}{\textbf{Remark: }}
\newcommand{\Question}{\textbf{Question: }}
\newcommand{\Extension}{\textbf{Extension: }}
\newcommand{\scp}[2]{\langle\,#1\,,\,#2\,\rangle} \newcommand{\scpp}{\langle\,\cdot\,,\,\cdot\,\rangle}


\usetheme{Madrid}
\setbeamertemplate{navigation symbols}{}

\addtobeamertemplate{frametitle}{}{
\begin{textblock*}{100mm}(0.85\textwidth,-1cm)
\includegraphics[height=1cm]{Figures/logo/logo.png}
\end{textblock*}}

\definecolor{themecolor}{RGB}{25,25,112} 

\usecolortheme[named=themecolor]{structure}

\setbeamertemplate{items}[default]

\hypersetup{
    colorlinks=true,
    linkcolor=themecolor,
    filecolor=themecolor,      
    urlcolor=themecolor,
    citecolor=themecolor,
}

\title{MATH285 / VV285 Integration Bee}
\subtitle{\textbf{Integration Battle}}
\institute[UM-SJTU JI]{University of Michigan-Shanghai Jiao Tong University Joint Institute}

\begin{document}

\begin{frame}
    \titlepage
    \begin{center}
        \includegraphics[height=2cm]{Figures/logo/logo2.png}
    \end{center}
\end{frame}

\subsection{Battle!}
\begin{frame}
    \frametitle{Question}
    \vfill
    \begin{Large}
        \begin{equation*}
            \int \sin(101x)(\sin x)^{99}\,dx
        \end{equation*}
    \end{Large}
    \vfill
\end{frame}

\begin{frame}
    \frametitle{Answer}
    \vfill
    \begin{Large}
        \begin{equation*}
            \frac{1}{100}\sin(100x)(\sin x)^{100}
        \end{equation*}    
    \end{Large}
    \vfill
\end{frame}

\begin{frame}
    \frametitle{Question}
    \vfill
    \begin{Large}
        \begin{equation*}
            \int \sqrt{x\sqrt{x\sqrt{x\cdots}}}\,dx
        \end{equation*}    
    \end{Large}
    \vfill
\end{frame}

\begin{frame}
    \frametitle{Answer}
    \vfill
    \begin{Large}
        \begin{equation*}
            \frac{x^2}{2}
        \end{equation*}    
    \end{Large}
    \vfill
\end{frame}

\begin{frame}
    \frametitle{Question}
    \vfill 
    \begin{Large}
        \begin{equation*}
            \int \frac{2x}{\sqrt{1 - x^4}}\,dx
        \end{equation*}
    \end{Large}
    \vfill
\end{frame}

\begin{frame}
    \frametitle{Answer}
    \vfill
    \begin{Large}
        \begin{equation*}
            \arcsin (x^2)
        \end{equation*}
    \end{Large}
    \vfill
\end{frame}

\begin{frame}
    \frametitle{Question}
    \vfill
    \begin{Large}
        \begin{equation*}
            \int x^3 e^{x^2}\,dx
        \end{equation*}    
    \end{Large}
    \vfill
\end{frame}

\begin{frame}
    \frametitle{Answer}
    \vfill
    \begin{Large}
        \begin{equation*}
            \frac{x^2 - 1}{2}e^{x^2}
        \end{equation*}    
    \end{Large}
    \vfill
\end{frame}

\begin{frame}
    \frametitle{Question}
    \vfill
    \begin{Large}
        \begin{equation*}
            \int^\infty_0 \frac{1}{e^x + 1}\,dx
        \end{equation*}    
    \end{Large}
    \vfill
\end{frame}

\begin{frame}
    \frametitle{Answer}
    \vfill
    \begin{Large}
        \begin{equation*}
            \log 2
        \end{equation*}    
    \end{Large}
    \vfill
\end{frame}

\begin{frame}
    \frametitle{Question}
    \vfill
    \begin{Large}
        \begin{equation*}
            \int \frac{2x+1}{2x^2+2x+1}\,dx
        \end{equation*}    
    \end{Large}
    \vfill
\end{frame}

\begin{frame}
    \frametitle{Answer}
    \vfill
    \begin{Large}
        \begin{equation*}
            \frac{1}{2}\ln(2x^2+2x+1)
        \end{equation*}    
    \end{Large}
    \vfill
\end{frame}

\end{document}

